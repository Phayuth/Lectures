\chapter{State Space Representation}
Linear State Space Form
\[
\dot{x} = A(t)x+Bu(t)
\]
Non-Linear State Space Form
\[
\dot{x} = f(t,x,u)
\]

\section{Forming State Space}
\begin{tcolorbox}[title=Process]
	\begin{itemize}
		\item Step 1 : Obtain Equation of Motion.
		\item Step 2 : Choose State Variables [ex: position, velocity ...].
		\item Step 3 : Take Derivative of State Vector.
		\item Step 4 : Write in State-Space form
		\item Step 5 : Write Output Equation.
	\end{itemize}
\end{tcolorbox}


\paragraph{Example 1} Obtain S.S from system below
\begin{itemize}
	\item Step 1 : Obtain Equation of Motion.
\end{itemize}
\[
\ddot{y} + 4 \dot{y} + 3 y = 3 u
\]
\begin{itemize}
	\item Step 2 : Choose State Variables. We would like to know \(y\) and \(\dot{y}\). Thus, Let Choose:
\end{itemize}
\[
\begin{split}
	X_1 &= y \\
	X_2 &= \dot{y}
\end{split}
\]
\begin{itemize}
	\item Step 3 : Take Derivative of State Vector.
\end{itemize}
\[
\begin{split}
	X_1 &= y => \dot{X}_1 = \dot{y}\\
	X_2 &= \dot{y} => \dot{X}_2 = \ddot{y} = 3u - 4 \dot{y} - 3 y
\end{split}
\]
\[
\begin{bmatrix}
	\dot{X}_1 \\
	\dot{X}_2 
\end{bmatrix} =
\begin{bmatrix}
	\dot{y}              \\
	3u - 4 \dot{y} - 3 y 
\end{bmatrix}
\]
\begin{itemize}
	\item Step 4 : Write in State-Space form.
\end{itemize}
\[
\dot{X} = 
\begin{bmatrix}
	\dot{X}_1 \\
	\dot{X}_2 
\end{bmatrix} =
\begin{bmatrix}
	0  &   & 1  \\
	-3 &   & -4 
\end{bmatrix}
\begin{bmatrix}
	X_1 \\
	X_2 
\end{bmatrix} +
\begin{bmatrix}
	0 \\
	3 
\end{bmatrix} u
\]
\begin{itemize}
	\item Step 5 : Write Output Equation. We choose \(y=y_{one}\) because we only interest in displacement only \(X_1\), if we are interested in velocity \(X_2\) as well we choose \(y=y_{two}\).
\end{itemize}
\[
y_{one} =
\begin{bmatrix}
	1 &   & 0 
\end{bmatrix}
\begin{bmatrix}
	X_1 \\
	X_2 
\end{bmatrix} \text{or  }
y_{two} = 
\begin{bmatrix}
	1 &   & 0 \\
	0 &   & 1 
\end{bmatrix}
\begin{bmatrix}
	X_1 \\
	X_2 
\end{bmatrix}
\]



\paragraph{Example 2} Obtain S.S from system of mass, spring, damper

\begin{figure}[ht]
	\centering
	%\def\svgscale{1}
	\includesvg{src/statespace/statespace_fig1}
\end{figure}

\begin{itemize}
	\item Step 1 : Obtain Equation of Motion. From the 2nd law of Newton:
\end{itemize}
\[
\sum \vec{F} = m\vec{a}
\]
\[
\begin{split}
	F - ky - c \dot{y} &= m \ddot{y} \\
	m \ddot{y} + c \dot{y} + ky &= F \\
	\ddot{y} + \frac{c}{m} \dot{y} + \frac{k}{m} y &= \frac{F}{m}
\end{split}
\]
\begin{itemize}
	\item Step 2 : Choose State Variables. We would like to know \(y\) and \(\dot{y}\). Thus, Let Choose:
\end{itemize}
\[
\begin{split}
	X_1 &= y \\
	X_2 &= \dot{y}
\end{split}
\]
\begin{itemize}
	\item Step 3 : Take Derivative of State Vector.
\end{itemize}
\[
\begin{split}
	X_1 &= y => \dot{X}_1 = \dot{y}\\
	X_2 &= \dot{y} => \dot{X}_2 = \ddot{y} = \frac{F}{m} - \frac{c}{m} \dot{y} - \frac{k}{m} y
\end{split}
\]
\[
\begin{bmatrix}
	\dot{X}_1 \\
	\dot{X}_2 
\end{bmatrix} =
\begin{bmatrix}
	\dot{y}                                           \\
	\frac{F}{m} - \frac{c}{m} \dot{y} - \frac{k}{m} y 
\end{bmatrix}
\]
\begin{itemize}
	\item Step 4 : Write in State-Space form.
\end{itemize}
\[
\dot{X} = 
\begin{bmatrix}
	\dot{X}_1 \\
	\dot{X}_2 
\end{bmatrix} =
\begin{bmatrix}
	0            &   & 1            \\
	\frac{-k}{m} &   & \frac{-c}{m} 
\end{bmatrix}
\begin{bmatrix}
	X_1 \\
	X_2 
\end{bmatrix} +
\begin{bmatrix}
	0           \\
	\frac{1}{m} 
\end{bmatrix} F
\]
\begin{itemize}
	\item Step 5 : Write Output Equation.
\end{itemize}
\[
y =
\begin{bmatrix}
	1 &   & 0 
\end{bmatrix}
\begin{bmatrix}
	X_1 \\
	X_2 
\end{bmatrix}
\]


\paragraph{Example 3} Obtain S.S from system of mass, spring with 2 vertical mass

\begin{figure}[ht]
	\centering
	%\def\svgscale{1}
	\includesvg{src/statespace/statespace_fig2}
\end{figure}


\begin{itemize}
	\item Step 1 : Obtain Equation of Motion. From the 2nd law of Newton:
\end{itemize}
\[
\sum \vec{F} = m\vec{a}
\]
\[
\begin{split}
	\text{Mass 1: } -k_1y_1 + k_2y_1 + u_1 + k_2y_2 = m_1\ddot{y}_1 \\
	\text{Mass 2: } -k_3y_2 - k_2y_2 + u_2 + k_2y_1 = m_2\ddot{y}_2
\end{split}
\]
\begin{itemize}
	\item Step 2 : Choose State Variables. We would like to know \(y\) and \(\dot{y}\). Thus, Let Choose:
\end{itemize}
\[
\begin{split}
	X_1 &= y_1 \\
	X_2 &= \dot{y}_1 \\
	X_3 &= y_2 \\
	X_4 &= \dot{y}_2
\end{split}
\]
\begin{itemize}
	\item Step 3 : Take Derivative of State Vector.
\end{itemize}
\[
\begin{split}
	X_1 &= y_1 => \dot{X}_1 = \dot{y}_1 \\
	X_2 &= \dot{y}_1 => \dot{X}_2 = \ddot{y}_1 = -\frac{k_1}{m_1}y_1 + \frac{k_2}{m_1}y_1 + \frac{1}{m_1}u_1 + \frac{k_2}{m_1}y_2 = \frac{k_2 - k_1}{m_1}y_1 + \frac{1}{m_1}u_1 + \frac{k_2}{m_1}y_2 \\
	X_3 &= y_2 => \dot{X}_3 = \dot{y}_2 \\ 
	X_4 &= \dot{y}_2 => \dot{X}_4 = \ddot{y}_2 = -\frac{k_3}{m_2}y_2 - \frac{k_2}{m_2}y_2 + \frac{1}{m_2}u_2 + \frac{k_2}{m_2}y_1 = \frac{-k_3 - k_2}{m_2}y_2 + \frac{1}{m_2}u_2 + \frac{k_2}{m_2}y_1
\end{split}
\]
\[
\begin{bmatrix}
	\dot{X}_1 \\
	\dot{X}_2 \\
	\dot{X}_3 \\
	\dot{X}_4 
\end{bmatrix} =
\begin{bmatrix}
	\dot{y}_1                                                         \\
	\frac{k_2 - k_1}{m_1}y_1 + \frac{1}{m_1}u_1 + \frac{k_2}{m_1}y_2  \\
	\dot{y}_2                                                         \\
	\frac{-k_3 - k_2}{m_2}y_2 + \frac{1}{m_2}u_2 + \frac{k_2}{m_2}y_1 
\end{bmatrix}
\]
\begin{itemize}
	\item Step 4 : Write in State-Space form.
\end{itemize}
\[
\dot{X} = 
\begin{bmatrix}
	\dot{X}_1 \\
	\dot{X}_2 \\
	\dot{X}_3 \\
	\dot{X}_4 
\end{bmatrix} =
\begin{bmatrix}
	0                     &   & 1 &   & 0                      &   & 0 \\
	\frac{k_2 - k_1}{m_1} &   & 0 &   & \frac{k_2}{m_1}        &   & 0 \\
	0                     &   & 0 &   & 0                      &   & 1 \\
	\frac{k_2}{m_2}       &   & 0 &   & \frac{-k_3 - k_2}{m_2} &   & 0 \\
\end{bmatrix}
\begin{bmatrix}
	X_1 \\
	X_2 \\
	X_3 \\
	X_4 
\end{bmatrix} +
\begin{bmatrix}
	0             &   & 0             \\
	\frac{1}{m_1} &   & 0             \\
	0             &   & 0             \\
	0             &   & \frac{1}{m_2} 
\end{bmatrix}
\begin{bmatrix}
	u_1 \\
	u_2 
\end{bmatrix}
\]
\begin{itemize}
	\item Step 5 : Write Output Equation.
\end{itemize}
\[
y =
\begin{bmatrix}
	1 &   & 0 &   & 0 &   & 0 \\ 
	0 &   & 0 &   & 1 &   & 0 \\ 
\end{bmatrix}
\begin{bmatrix}
	X_1 \\
	X_2 \\
	X_3 \\
	X_4 
\end{bmatrix}
\]

\paragraph{Example 4} Solve system of single mass and spring and force using Matlab. 
\begin{lstlisting}[language=MATLAB,title=MATLAB Numerical Method using ode45(Runge Kutta)]
	[t,x] = ode45(@f,tspan,x_0)
	t = time
	x = state vector
	ode45 = solver
	f = function
	tspan = t_0 -> t_f
	x_0 = initial condition
	
	
	Example:
	
	tspan = [0,10];
	x_0 = [0,0];
	
	function dx = model(t,x)
	% dx = Ax+Bu
	k = 0.01;m=1;u=2;
	A = [0 1;-k/m 0];
	B = [0;1/m];
	dx = A*x + B*u;
	
	[t,x] = ode45(@model,tspan,x_0);
	plot(t,x(:;1))
	hold on
	plot(t,x(:;2))
	legend('displacement','velocity')
\end{lstlisting}

\paragraph{Example 5} Obtain S.S from system of mass, spring, damper with 2 horizontal mass

\begin{figure}[ht]
	\centering
	%\def\svgscale{1}
	\includesvg{src/statespace/statespace_fig2}
\end{figure}

Equation of Motion
\[
\sum \vec{F} = m\vec{a}
\]
\[
\begin{split}
	\text{Mass 1: }& m_1\ddot{p}(t) + b_1\dot{p}(t) + k_1p(t) = u(t) +k_1q(t)+b_1\dot{q}(t) \\
	\text{Mass 2: }& m_2\ddot{q}(t) + (k_1+k_2)q(t) + (b_1+b_2)\dot{q}(t) = k_1p(t)+b_1\dot{p}(t) \\
	&\ddot{p}(t)= \frac{1}{m_1}u(t) +\frac{k_1}{m_1}q(t)+\frac{b_1}{m_1}\dot{q}(t) - \frac{b_1}{m_1}\dot{p}(t) - \frac{k_1}{m_1}p(t)\\
	&\ddot{q}(t)= \frac{k_1}{m_2}p(t)+\frac{b_1}{m_2}\dot{p}(t) - \frac{(k_1+k_2)}{m_2}q(t) - \frac{(b_1+b_2)}{m_2}\dot{q}(t)
\end{split}
\]
Let:
\[
x = 
\begin{bmatrix}
	x_1 \\
	x_2 \\
	x_3 \\
	x_4 
\end{bmatrix}
= 
\begin{bmatrix}
	p       \\
	q       \\
	\dot{p} \\
	\dot{q} 
\end{bmatrix}
=>
\dot{x} = 
\begin{bmatrix}
	\dot{x}_1 \\
	\dot{x}_2 \\
	\dot{x}_3 \\
	\dot{x}_4 
\end{bmatrix}
= 
\begin{bmatrix}
	\dot{p}  \\
	\dot{q}  \\
	\ddot{p} \\
	\ddot{q} 
\end{bmatrix}
\]
Thus, we get state space form:
\[
\dot{x} = 
\begin{bmatrix}
	\dot{x}_1 \\
	\dot{x}_2 \\
	\dot{x}_3 \\
	\dot{x}_4 
\end{bmatrix} =
\begin{bmatrix}
	0                 &   & 0                       &   & 1                 &   & 0                       \\
	0                 &   & 0                       &   & 0                 &   & 1                       \\
	- \frac{k_1}{m_1} &   & \frac{k_1}{m_1}         &   & - \frac{b_1}{m_1} &   & \frac{b_1}{m_1}         \\
	\frac{k_1}{m_2}   &   & - \frac{(k_1+k_2)}{m_2} &   & \frac{b_1}{m_2}   &   & - \frac{(b_1+b_2)}{m_2} 
\end{bmatrix}
\begin{bmatrix}
	x_1 \\
	x_2 \\
	x_3 \\
	x_4 
\end{bmatrix} +
\begin{bmatrix}
	0             \\
	0             \\
	\frac{1}{m_1} \\
	0             
\end{bmatrix} u(t)
\]

\[
y =
\begin{bmatrix}
	1 &   & 0 &   & 0 &   & 0 
\end{bmatrix}
\begin{bmatrix}
	x_1 \\
	x_2 \\
	x_3 \\
	x_4 
\end{bmatrix}
\]




\section{State Space of Scalar Differential Equation System}

\subsection{Case 1}
Consider equation below:
\[
y^{(n)} + a_1y^{(n-1)} + ... + a_{n-1}y' + a_n y = u  \leftarrow \text{Input has not derivative}
\]
Let:
\[
x = 
\begin{bmatrix}
	x_1     \\
	x_2     \\
	\vdots  \\
	x_{n-1} \\
	x_n     
\end{bmatrix}
=
\begin{bmatrix}
	y       \\
	y'      \\
	\vdots  \\
	y^{n-1} \\
	y^n     
\end{bmatrix}
\]
Thus
\[
\dot{x} = 
\begin{bmatrix}
	\dot{x}_1     \\
	\dot{x}_2     \\
	\vdots        \\
	\dot{x}_{n-1} \\
	\dot{x}_n     
\end{bmatrix}
=
\begin{bmatrix}
	y'                                \\
	y''                               \\
	\vdots                            \\
	y^n                               \\
	-a_0x_1 - a_1x_2 ... - a_nx_n + u 
\end{bmatrix}
=
\begin{bmatrix}
	\dot{x}_2                         \\
	\dot{x}_3                         \\
	\vdots                            \\
	\dot{x}_n                         \\
	-a_0x_1 - a_1x_2 ... - a_nx_n + u 
\end{bmatrix}
\]
Arrange into SS form:
\[
\dot{x} = 
\begin{bmatrix}
	0      &   & 1        &   & 0        &   & ... &   & 0      \\
	0      &   & 0        &   & 1        &   & ... &   & 0      \\
	\vdots &   & \vdots   &   & \vdots   &   & ... &   & \vdots \\
	-a_n   &   & -a_{n-1} &   & -a_{n-2} &   & ... &   & -a_1   
\end{bmatrix}
\begin{bmatrix}
	x_1    \\
	x_2    \\
	\vdots \\
	x_n    
\end{bmatrix} +
\begin{bmatrix}
	0      \\
	0      \\
	\vdots \\
	1      
\end{bmatrix} u
\]
\[
y =
\begin{bmatrix}
	1 &   & 0 &   & ... &   & 0 
\end{bmatrix}
\begin{bmatrix}
	x_1    \\
	x_2    \\
	\vdots \\
	x_n    
\end{bmatrix}
\]
We have a corresponding Transfer Function is 
\[\frac{Y(s)}{U(s)} = \frac{1}{s^n+a_1s^{n-1}+...+a_{n-1}s+a_n}\]

\subsection{Case 2}
Consider equation below:
\[
y^{(n)} + a_1y^{(n-1)} + ... + a_{n-1}y' + a_n y = \beta_0u^n + \beta_1u^{n-1}+ ... +\beta_nu  \leftarrow \text{Input has derivative}
\]
Let:
\[
\begin{split}
	x_1 &= y - \beta_0 u \\
	x_2 &= y' - \beta_0 u' - \beta_1 u = x'_1-\beta_1u \\
	\vdots \\
	x_n &= y^{n-1} - \beta_0 u^{n-1} - ... - \beta_{n-1} u = x'_{n-1}-\beta_{n-1}u
\end{split}
\]
Where \(\beta_0 , \beta_1 , ... , \beta_{n-1}\) are determined from:
\[
\begin{split}
	\beta_0 &= b_0 \\
	\beta_1 &= b_1 - a_1\beta_0 \\
	\beta_2 &= b_2 - a_1\beta_1 - a_2\beta_0 \\
	\vdots
\end{split}
\]
Arrange into SS form:
\[
\dot{x} = 
\begin{bmatrix}
	0      &   & 1        &   & 0        &   & ... &   & 0      \\
	0      &   & 0        &   & 1        &   & ... &   & 0      \\
	\vdots &   & \vdots   &   & \vdots   &   & ... &   & \vdots \\
	-a_n   &   & -a_{n-1} &   & -a_{n-2} &   & ... &   & -a_1   
\end{bmatrix}
\begin{bmatrix}
	x_1    \\
	x_2    \\
	\vdots \\
	x_n    
\end{bmatrix} +
\begin{bmatrix}
	\beta_1 \\
	\beta_2 \\
	\vdots  \\
	\beta_n 
\end{bmatrix} u
\]
\[
y =
\begin{bmatrix}
	1 &   & 0 &   & ... &   & 0 
\end{bmatrix}
\begin{bmatrix}
	x_1    \\
	x_2    \\
	\vdots \\
	x_n    
\end{bmatrix} + \beta_0 u
\]
We have a corresponding Transfer Function is 
\[\frac{Y(s)}{U(s)} = \frac{b_0 s^n + b_1s^{n-1} + ... + b_{n-1}s+ b_n}{s^n+a_1s^{n-1}+...+a_{n-1}s+a_n}\]


\section{Transfer Function to State Space}
\paragraph{Example}
\[\frac{Y(s)}{U(s)} = \frac{100}{s^4 + 20s^3 + 10s^2 + 7s + 100}\]
\[(s^4 + 20s^3 + 10s^2 + 7s + 100)Y(s) = 100 U(s)\]
Taking Inverse Laplace Transform
\[y^{(4)} + 20y^{(3)} + 10 y'' + 7y' + 100y = 100 u\]
Let:
\[
\begin{split}
	x_1 &= y   => \dot{x}_1 = y' = x_2\\
	x_2 &= y'  => \dot{x}_2 = y'' = x_3\\
	x_3 &= y'' => \dot{x}_3 = y^{(3)} = x_4\\
	x_3 &= y'''=> \dot{x}_4 = y^{(4)} = 100u - 20y^{(3)} - 10 y'' - 7y' - 100y
\end{split}
\]
State Space form:
\[
\dot{x} = 
\begin{bmatrix}
	0    &   & 1  &   & 0   &   & 0   \\
	0    &   & 0  &   & 1   &   & 0   \\
	0    &   & 0  &   & 0   &   & 1   \\
	-100 &   & -7 &   & -10 &   & -20 
\end{bmatrix}
\begin{bmatrix}
	x_1 \\
	x_2 \\
	x_3 \\
	x_4 
\end{bmatrix} + 
\begin{bmatrix}
	0   \\
	0   \\
	0   \\
	100 
\end{bmatrix} u
\]
\[
y =
\begin{bmatrix}
	1 &   & 0 &   & 0 &   & 0 
\end{bmatrix}
\begin{bmatrix}
	x_1 \\
	x_2 \\
	x_3 \\
	x_4 
\end{bmatrix}
\]


\section{State Space to Transfer Function}
We have a Transfer Function:
\[
\frac{Y(s)}{U(s)} = G(s)
\]
with state space in form of:
\[
\begin{split}
	\dot{x} &= Ax + Bu \\
	y &= Cx + Du
\end{split}
\]
Let have a Laplace transform of SS:
\[
\begin{split}
	sX(s)-x(0) &= AX(s) + BU(s) \\
	Y(s) &= CX(s) + DU(s)
\end{split}
\]
Assuming \(x(0) = 0 IC\), we get:
\[
\begin{split}
	sX(s) - AX(s) &= BU(s) \\
	(sI - A)X(s) &= BU(s) \\
	(sI - A)^{-1}(sI - A)X(s) &= (sI - A)^{-1}BU(s) \\
	X(s) &= (sI - A)^{-1}BU(s)
\end{split}
\]
Substitute into \(Y(s)\)
\[
\begin{split}
	Y(s) &= C[(sI - A)^{-1}BU(s)] + DU(s) \\
	Y(s) &= C(sI - A)^{-1}BU(s) + DU(s) \\
	Y(s) &= [C(sI - A)^{-1}B + D]U(s)
\end{split}
\]
Thus the Transfer function can be found by:
\[
G(s) = C(sI - A)^{-1}B + D
\]


\paragraph{Example}
\[
\dot{x} = 
\begin{bmatrix}
	0  &   & 1   &   & 0  \\
	0  &   & 0   &   & 1  \\
	-5 &   & -25 &   & -5 
\end{bmatrix}
\begin{bmatrix}
	x_1 \\
	x_2 \\
	x_3 
\end{bmatrix} + 
\begin{bmatrix}
	0    \\
	25   \\
	-120 
\end{bmatrix} u
\]
\[
y =
\begin{bmatrix}
	1 &   & 0 &   & 0 
\end{bmatrix}
\begin{bmatrix}
	x_1 \\
	x_2 \\
	x_3 
\end{bmatrix}
\]

\[
G(s) = 
\begin{bmatrix}
	1 &   & 0 &   & 0 
\end{bmatrix}
[
\begin{bmatrix}
	s &   & 0 &   & 0 \\
	0 &   & s &   & 0 \\
	0 &   & 0 &   & s 
\end{bmatrix}
-
\begin{bmatrix}
	0  &   & 1   &   & 0  \\
	0  &   & 0   &   & 1  \\
	-5 &   & -25 &   & -5 
\end{bmatrix}]^{-1}
\begin{bmatrix}
	0    \\
	25   \\
	-120 
\end{bmatrix} + 0
\]

\[
G(s) = 
\begin{bmatrix}
	1 &   & 0 &   & 0 
\end{bmatrix}
\begin{bmatrix}
	s  &   & 1   &   & 0   \\
	0  &   & s   &   & 1   \\
	-5 &   & -25 &   & s+5 
\end{bmatrix}^{-1}
\begin{bmatrix}
	0    \\
	25   \\
	-120 
\end{bmatrix}
\]

\[
G(s) = 
\begin{bmatrix}
	1 &   & 0 &   & 0 
\end{bmatrix}
\begin{bmatrix}
	\frac{(s^2+5s+25)}{(s^3+5s^2+25s-5)} &   & \frac{(-s-5)}{(s^3+5s^2+25s-5)}   &   & \frac{1}{(s^3+5s^2+25s-5)}   \\
	\frac{-5}{(s^3+5s^2+25s-5)}          &   & \frac{(s^2+5s)}{(s^3+5s^2+25s-5)} &   & \frac{-s}{(s^3+5s^2+25s-5)}  \\
	\frac{5s}{(s^3+5s^2+25s-5)}          &   & \frac{(25s-5)}{(s^3+5s^2+25s-5)}  &   & \frac{s^2}{(s^3+5s^2+25s-5)} 
\end{bmatrix}
\begin{bmatrix}
	0    \\
	25   \\
	-120 
\end{bmatrix}
\]

\[
G(s) = 
\begin{bmatrix}
	1 &   & 0 &   & 0 
\end{bmatrix}
\begin{bmatrix}
	\frac{(-25s-245)}{(s^3+5s^2+25s-5)}         \\
	\frac{(25s^2+245s)}{(s^3+5s^2+25s-5)}       \\
	\frac{(-120s^2+625s-125)}{(s^3+5s^2+25s-5)} 
\end{bmatrix}
\]

\[
G(s) = \frac{(-25s-245)}{(s^3+5s^2+25s-5)}
\]


Thus
\[
G(s) = \frac{25s + 245}{s^3 + 5s^2 + 25s + 5}
\]