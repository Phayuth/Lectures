\documentclass[12pt,a4paper]{article}
\usepackage[utf8]{inputenc}
\usepackage[T1]{fontenc}
\usepackage{amsmath}
\usepackage{amsfonts}
\usepackage{amssymb}
\usepackage{graphicx}
\usepackage[left=2.54cm, right=2.54cm, top=2.54cm, bottom=2.54cm]{geometry}
\usepackage[hidelinks]{hyperref} %for reference automatically
\usepackage{fancyhdr}

\usepackage{tikz}
\usetikzlibrary{positioning}

\pagestyle{fancy} % for header and footer
\fancyhf{}
\fancyhead[LE,RO]{Prepared by: Phayuth}
\fancyhead[RE,LO]{Supervisor : Dr.Sarot Srang}
\fancyfoot[LE,RO]{Page \thepage}
\renewcommand{\headrulewidth}{2pt}
\renewcommand{\footrulewidth}{1pt} % for header and footer

% THIS IS THE XML CODE INCLUDE==============================================================
\usepackage{listings}
\usepackage{xcolor}
\definecolor{codegreen}{rgb}{0,0.6,0}
\definecolor{codegray}{rgb}{0.5,0.5,0.5}
\definecolor{codepurple}{rgb}{0.58,0,0.82}
\definecolor{backcolour}{rgb}{0.95,0.95,0.92}
\lstset{
	backgroundcolor=\color{backcolour},   
	commentstyle=\color{codegreen},
	keywordstyle=\color{magenta},
	numberstyle=\tiny\color{codegray},
	stringstyle=\color{codepurple},
	numbers=left,
	breaklines=true,
	tabsize=2,
	basicstyle=\ttfamily\footnotesize,
	literate={\ \ }{{\ }}1
}


\begin{document}
	\section*{\centering Foundation - Lesson 4 : 2nd Order ODE Standard Form Natural Frequency and Damping Ratio}
	%	\begin{itemize}
		%		\item {\makebox[1cm]{\(x(t)\)\hfill} is Input of the system}
		%		\item {\makebox[1cm]{\(y(t)\)\hfill} is Output of the system}
		%		\item {\makebox[1cm]{\(h(t)\)\hfill} is the system}
		%	\end{itemize}
	
	%	\begin{equation}
		%		\boxed{
			%			H(s) = \frac{Y(s)}{X(s)}
			%		}
		%		\label{eq3}
		%	\end{equation}
	%   \section{Background}
	\section{Spring Mass Damper Modeling}
	(Free Body Diagram)
	
	\[
	\begin{split}
		-kx(t) - b\dot{x}(t) &= m\ddot{x}(t) \\
		m\ddot{x}(t) + kx(t) + b\dot{x}(t)&= 0 \\
		\ddot{x} + \frac{b}{m} \dot{x}+\frac{k}{m} x(t) &= 0
	\end{split}
	\]
	Let have the differential model above to look like the General Standard Form of :
	\[
	\boxed{\ddot{x}(t)+2\zeta\omega_n\dot{x}(t)+\omega_n^2x(t)=0}
	\]
	Where :
	\begin{itemize}
		\item $ \zeta $ is damping ratio
		\item $ \omega_n $ is natural frequency
	\end{itemize}
	Thus, we have :
	\begin{itemize}
		\item $ \frac{b}{m} = 2\zeta\omega_n \rightarrow \zeta = \frac{b}{2\sqrt{km}}$ 
		\item $ \frac{k}{m} = \omega_n^2 \rightarrow \omega_n = \sqrt{\frac{k}{m}}$
	\end{itemize}
	Let solve the Standard Form $ x(t) $:
	\[ 
	\ddot{x}(t)+2\zeta\omega_n\dot{x}(t)+\omega_n^2x(t)=0 
	\]
	Using Laplace Transform :
	\[ 
	\begin{split}
		s^2 X(s) - sx(0) -  \dot{x}(0) + 2\zeta\omega_n(sX(s)-x(0)) + \omega_n^2X(s) = 0 \\
		s^2 X(s) - sx_0 -  \dot{x}_0 + 2\zeta\omega_n(sX(s)-x_0) + \omega_n^2X(s) = 0 \\
	\end{split}
	\]
	\[ 
	X(s) = \frac{sx_0 + \dot{x}_0 + 2\zeta\omega_nx_0}{s^2 + 2\zeta\omega_ns+\omega_n^2}
	\]
	Let Find the root of the Denominator of $ X(s) $. From solving the 2nd order quadratic formula, we have the root :
	\[ 
	s_{1,2} = \frac{-2\zeta\omega_n \pm \sqrt{(2\zeta\omega_n)^2 - 4\omega_n^2}}{2} = -\zeta\omega_n \pm \omega_n\sqrt{\zeta^2-1}
	\]
	From the root, we can see that there are 3 cases:
	\begin{itemize}
		\item Distinct Real Root
		\item Double Real Root
		\item Complex Root
	\end{itemize}
	
	
	\subsection{Distinct Real Roots}
	To have the Distinct Real Root Case, We need:
	\[ 
	\begin{split}
		(2\zeta\omega_n)^2 - 4\omega_n^2 &> 0 \\
		4\zeta^2\omega_n^2 - 4\omega_n^2 &> 0 \\
		4\omega_n^2 (\zeta^2 - 1) &> 0 \\
		(\zeta^2 - 1) &> 0 \\
		\zeta^2  &> 1 \\
		\zeta &> 1
	\end{split}
	\]
	We get Over-damped Case from the damping ratio of $ \boxed{\zeta > 1} $
	
	
	\subsection{Double Real Root}
	To have the Double Real Root Case, We need:
	\[ 
	\begin{split}
		(2\zeta\omega_n)^2 - 4\omega_n^2 &= 0 \\
		4\zeta^2\omega_n^2 - 4\omega_n^2 &= 0 \\
		4\omega_n^2 (\zeta^2 - 1) &= 0 \\
		(\zeta^2 - 1) &= 0 \\
		\zeta^2  &= 1 \\
		\zeta &= 1
	\end{split}
	\]
	We get Critically damped Case from the damping ratio of $ \boxed{\zeta = 1} $ \\
	From the mathematical perspective, the damping ratio is unity (1) mean Critically damped. Where some people from control perspective prefer the damping ratio of $ \frac{1}{\sqrt{2}} $ to be Critically damped.
	
	
	\subsection{Complex Root}
	To have the Complex Root Case, We need:
	\[ 
	\begin{split}
		(2\zeta\omega_n)^2 - 4\omega_n^2 &< 0 \\
		4\zeta^2\omega_n^2 - 4\omega_n^2 &< 0 \\
		4\omega_n^2 (\zeta^2 - 1) &< 0 \\
		(\zeta^2 - 1) &< 0 \\
		\zeta^2  &< 1 \\
		\zeta &< 1
	\end{split}
	\]
	
	\section{Discussion of Each Cases}
	\subsection{Over-damped Case ($ \zeta > 1$)}
	Above equation can be written as:
	\[ 
	X(s) = \frac{sx_0 + \dot{x}_0 + 2\zeta\omega_nx_0}{s^2 + 2\zeta\omega_ns+\omega_n^2} = \frac{a_1}{s+r_1}+\frac{a_2}{s+r_2}
	\]
	After using Partial Fraction Decomposition, we get:
	\[ 
	a_1 = \frac{-\dot{x}_0 + x_0 (-\zeta \omega_n + \sqrt{(\zeta^2-1)\omega_n^2})}{2\sqrt{(\zeta^2-1)\omega_n^2}}
	\]
	\[ 
	a_2 = \frac{\dot{x}_0 + x_0 (\zeta \omega_n + \sqrt{(\zeta^2-1)\omega_n^2})}{2\sqrt{(\zeta^2-1)\omega_n^2}}
	\]
	Thus the solution of differential equation $ \ddot{x}(t)+2\zeta\omega_n\dot{x}(t)+\omega_n^2x(t)=0  $ where $ x(0)=x_0, \dot{x}(0) = \dot{x}_0 $ is :
	\[ 
	x(t) = a_1e^{r_1t}+a_2e^{r_2t} 
	\]
	Where:
	\[
	\begin{split}
		a_1 &= \frac{-\dot{x}_0 + x_0 (-\zeta \omega_n + \sqrt{(\zeta^2-1)\omega_n^2})}{2\sqrt{(\zeta^2-1)\omega_n^2}} \\
		a_2 &= \frac{\dot{x}_0 + x_0 (\zeta \omega_n + \sqrt{(\zeta^2-1)\omega_n^2})}{2\sqrt{(\zeta^2-1)\omega_n^2}} \\
		r_1,r_2 &= -\zeta\omega_n \pm \omega_n\sqrt{\zeta^2-1}
	\end{split} 
	\]
	
	\subsection{Critically damped Case ($ \zeta = 1$)}
	We have the root : $ r_1,r_2 = -\zeta\omega_n \pm \omega_n\sqrt{\zeta^2-1} $\\
	By substitute $ \zeta = 1 $, we get the root : 
	\[ r_1,r_2 = -\omega_n \]
	Above equation can be written as:
	\[ 
	X(s) = \frac{sx_0 + \dot{x}_0 + 2\zeta\omega_nx_0}{s^2 + 2\zeta\omega_ns+\omega_n^2} = \frac{a_1}{s+\omega_n}+\frac{a_2}{(s+\omega_n)^2}
	\]
	After using Partial Fraction Decomposition, we get:
	\[ 
	a_1 = x_0
	\]
	\[ 
	a_2 = \dot{x}_0 + x_0\omega_n
	\]
	Thus the solution of differential equation $ \ddot{x}(t)+2\zeta\omega_n\dot{x}(t)+\omega_n^2x(t)=0  $ where $ x(0)=x_0, \dot{x}(0) = \dot{x}_0 $ is :
	\[ 
	x(t) = x_0e^{-\omega_nt}+te^{-\omega_nt}(\dot{x}_0+x_0\omega_n)
	\]
	Where:
	\[
	\begin{split}
		a_1 &= x_0\\
		a_2 &= \dot{x}_0 + x_0\omega_n
	\end{split} 
	\]
	
	
	\subsection{Under damped Case ($ \zeta < 1$)}
	We have the root : $ r_1,r_2 = -\zeta\omega_n \pm \omega_n\sqrt{\zeta^2-1} $\\
	By modify the square root part, we get :
	\[
	\begin{split}
		r_1,r_2 &= -\zeta\omega_n \pm \omega_n\sqrt{-1(1-\zeta^2)} \\
		&= -\zeta\omega_n \pm \omega_n\sqrt{(1-\zeta^2)}\sqrt{-1} \\
		&= -\zeta\omega_n \pm \omega_n\sqrt{(1-\zeta^2)}i \\
	\end{split}
	\]
	Let:
	\[
	\begin{split}
		\sigma &= \zeta\omega_n\\
		\omega_d &=\omega_n\sqrt{(1-\zeta^2)} \\
	\end{split}
	\]
	We can write the root as :
	\[
	r_1,r_2 = -\sigma \pm \omega_d i
	\]
	Rewrite the root in form of :
	\[
	s^2 + 2\zeta\omega_ns+\omega_n^2 = (s+\alpha)^2+w_n^2
	\]
	We get :
	\[
	\begin{split}
		\alpha &= \zeta\omega_n\\
		\omega &= \sqrt{(1-\zeta^2)\omega_n^2} \\
	\end{split}
	\]
	Above equation can be written as:
	\[ 
	X(s) = \frac{sx_0 + \dot{x}_0 + 2\zeta\omega_nx_0}{s^2 + 2\zeta\omega_ns+\omega_n^2} = \frac{sx_0 + \dot{x}_0 + 2\zeta\omega_nx_0}{(s+\alpha)^2+w_n^2}
	\]
	After using Partial Fraction Decomposition, The solution of differential equation $ \ddot{x}(t)+2\zeta\omega_n\dot{x}(t)+\omega_n^2x(t)=0  $ where $ x(0)=x_0, \dot{x}(0) = \dot{x}_0 $ is :
	\[
	x(t) = a cos(\omega t) + b sin(\omega t)
	\]
	Where:
	\[
	\begin{split}
		a &= x_0e^{-\alpha t}\\
		b &= \frac{\dot{x}_0 + x_0\zeta\omega_n}{\omega} e^{-\alpha t}\\
		\alpha &= \zeta\omega_n \\
		\omega &= \sqrt{(1-\zeta^2)\omega_n^2} \\
	\end{split} 
	\]
	Or in simple form of :
	\[
	x(t) = A e^{-\zeta \omega_n t}cos(\omega_d t - \phi)
	\]
	Where:
	\[
	\begin{split}
		A &= \sqrt{x_0^2 + \frac{(\dot{x}_0 + x_0\zeta\omega_n)^2}{(1-\zeta^2)\omega_n^2}}\\
		\omega_d &= \omega_n\sqrt{(1-\zeta^2)} \\
		\phi & = atan2(\frac{\dot{x}_0 + x_0\zeta\omega_n}{\omega_n\sqrt{(1-\zeta^2)}},x_0)
	\end{split} 
	\]
\end{document}