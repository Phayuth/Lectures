\documentclass[12pt,a4paper]{article}
\usepackage[utf8]{inputenc}
\usepackage[T1]{fontenc}
\usepackage{amsmath}
\usepackage{amsfonts}
\usepackage{amssymb}
\usepackage{graphicx}
\usepackage[left=2.54cm, right=2.54cm, top=2.54cm, bottom=2.54cm]{geometry}
\usepackage[hidelinks]{hyperref} %for reference automatically
\usepackage{fancyhdr}

\pagestyle{fancy} % for header and footer
\fancyhf{}
\fancyhead[LE,RO]{Prepared by: Phayuth}
\fancyhead[RE,LO]{Supervisor : Dr.Sarot Srang}
\fancyfoot[LE,RO]{Page \thepage}
\renewcommand{\headrulewidth}{2pt}
\renewcommand{\footrulewidth}{1pt} % for header and footer

% THIS IS THE XML CODE INCLUDE==============================================================
\usepackage{listings}
\usepackage{xcolor}
\definecolor{codegreen}{rgb}{0,0.6,0}
\definecolor{codegray}{rgb}{0.5,0.5,0.5}
\definecolor{codepurple}{rgb}{0.58,0,0.82}
\definecolor{backcolour}{rgb}{0.95,0.95,0.92}
\lstset{
	backgroundcolor=\color{backcolour},   
	commentstyle=\color{codegreen},
	keywordstyle=\color{magenta},
	numberstyle=\tiny\color{codegray},
	stringstyle=\color{codepurple},
	numbers=left,
	breaklines=true,
	tabsize=2,
	basicstyle=\ttfamily\footnotesize,
	literate={\ \ }{{\ }}1
}


\begin{document}
	\section*{\centering Lesson 3 : DC Motor Lamped Parameters Identification}
	\section{Background}
	DC Motor is widely used in many applications such as robot, industrial application -etc. It has been produced in great number, some is at high standard with larger documentation and specification while some are inexpensive with little to none of documentation. To be able to use the dc motor efficiently, we must know it mathematical model. In this lesson, we use a variant of a famous algorithm known as Extended Kalman Filter (EKF) and Unscented Kalman Filter (UKF) to estimate the dc motor model.\par
	
	\section{DC Motor Stochastic State Space Model}
	From Lecture 1 : DC Motor, We have a mathematical model to represent the motor:\\
	Model with neglect the coulomb friction:
	\[\boxed{\dot{\omega}(t) = - a\omega(t) + bv_a(t)}\]
	Model with the coulomb friction:
	\[\boxed{\dot{\omega}(t) = - a\omega(t) + bv_a(t) - csign(\omega(t))}\]
	
	\subsection{Model with neglect the coulomb friction}
	From the model:
	\begin{equation}
		\dot{\omega}(t) = - a \omega(t) + b v_a(t)
		\label{eq1}
	\end{equation}
	In control system, we have a state space model for a nonlinear model:
	\[\dot{x}(t) = f(t,x(t),u(t)) + v_{noise}(t)\]
	\[y(t) = h(t,x(t),u(t)) + \omega_{noise}(t)\]
	Where:
	\begin{itemize}
		\item {\makebox[1cm]{\(\dot{x}(t)\)\hfill} is rate of change of state }
		\item {\makebox[1cm]{\(x(t)\)\hfill} is the current state}
		\item {\makebox[1cm]{\(u(t)\)\hfill} is the input to the system}
		\item {\makebox[1cm]{\(y(t)\)\hfill} is the measurement model}
		\item {\makebox[1.5cm]{\(v_{noise}(t)\)\hfill} is random process noise}
		\item {\makebox[1.5cm]{\(\omega_{noise}(t)\)\hfill} is random measurement noise}
	\end{itemize}
	From \autoref{eq1}, Let:
	\begin{equation}
		\begin{split}
			x_1 = \omega\\
			x_2 = a\\
			x_3 = b
		\end{split}
		\label{eq2}
	\end{equation}
	We get:
	\begin{equation}
		\begin{split}
			\dot{x_1} = \dot{\omega} = - a \omega(t) + b v_a(t) &= -x_2 x_1 + x_3 v_a(t)\\
			\dot{x_2} &= 0 \\
			\dot{x_3} &= 0
		\end{split}
		\label{eq3}
	\end{equation}
	In continuous nonlinear stochastic system matrix form:
	\begin{equation}
		\boxed{\begin{split}
				\dot{x}(t) =
				\begin{bmatrix}
					\dot{x_1} \\
					\dot{x_2} \\
					\dot{x_3} 
				\end{bmatrix}(t) &=
				\begin{bmatrix}
					-x_2 x_1 + x_3 v_a(t) \\
					0                     \\
					0                     
				\end{bmatrix} +\sqrt{Q_c}v_{noise}(t)\\
				y(t) &= 
				\begin{bmatrix}
					1 & 0 & 0 
				\end{bmatrix}
				\begin{bmatrix}
					x_1 \\
					x_2 \\
					x_3 
				\end{bmatrix}(t)+\sqrt{R}\omega_{noise}(t)
		\end{split}}
		\label{eq4}
	\end{equation}
	Discretize the continuous model from \autoref{eq4}:
	\begin{equation}
		\begin{split}
			\dot{x}(t) &=
			\begin{bmatrix}
				-x_2 x_1 + x_3 v_a(t) \\
				0                     \\
				0                     
			\end{bmatrix} +\sqrt{Q_c}v_{noise}(t)\\
			y(t) &= 
			\begin{bmatrix}
				1 & 0 & 0 
			\end{bmatrix}
			\begin{bmatrix}
				x_1 \\
				x_2 \\
				x_3 
			\end{bmatrix}(t)+\sqrt{R}\omega_{noise}(t)
		\end{split}
		\label{eq5}
	\end{equation}
	\begin{equation}
		\begin{split}
			\frac{x_{k+1} - x_k}{T_s} &=
			\begin{bmatrix}
				-x_2 x_1 + x_3 v_a(t) \\
				0                     \\
				0                     
			\end{bmatrix} +\sqrt{Q_c}v_{noise}(t)\\
			y_k &= 
			\begin{bmatrix}
				1 & 0 & 0 
			\end{bmatrix}
			\begin{bmatrix}
				x_1 \\
				x_2 \\
				x_3 
			\end{bmatrix}_k+\sqrt{R}\omega_{noise}(t)
		\end{split}
		\label{eq6}
	\end{equation}
	We get a discretized nonlinear stochastic system in matrix form:
	\begin{equation}
		\boxed{\begin{split}
				x_{k+1} &= x_k + T_s
				\begin{bmatrix}
					-x_2 x_1 + x_3 v_{a,k} \\
					0                      \\
					0                      
				\end{bmatrix} +\sqrt{T_s Q_d}v_{noise,k}\\
				y_k &= 
				\begin{bmatrix}
					1 & 0 & 0 
				\end{bmatrix}
				\begin{bmatrix}
					x_1 \\
					x_2 \\
					x_3 
				\end{bmatrix}_k+\sqrt{R}\omega_{noise,k}
		\end{split}}
		\label{eq7}
	\end{equation}
	\subsection{Model the coulomb friction}
	From the model:
	\begin{equation}
		\dot{\omega}(t) = - a \omega(t) + b v_a(t) - csign(\omega(t))
		\label{eq8}
	\end{equation}
	From \autoref{eq8}, Let:
	\begin{equation}
		\begin{split}
			x_1 = \omega\\
			x_2 = a\\
			x_3 = b\\
			x_4 = c
		\end{split}
		\label{eq9}
	\end{equation}
	We get:
	\begin{equation}
		\begin{split}
			\dot{x_1} = \dot{\omega} = - a \omega(t) + b v_a(t) &= -x_2 x_1 + x_3 v_a(t) - x_4sign(x_1)  \\
			\dot{x_2} &= 0 \\
			\dot{x_3} &= 0 \\
			\dot{x_4} &= 0
		\end{split}
		\label{eq10}
	\end{equation}
	In continuous nonlinear stochastic system matrix form:
	\begin{equation}
		\boxed{\begin{split}
				\dot{x}(t) =
				\begin{bmatrix}
					\dot{x_1} \\
					\dot{x_2} \\
					\dot{x_3} \\
					\dot{x_4} 
				\end{bmatrix}(t) &=
				\begin{bmatrix}
					-x_2 x_1 + x_3 v_a(t) - x_4sign(x_1) \\
					0                                    \\
					0                                    \\
					0                                    
				\end{bmatrix} +\sqrt{Q_c}v_{noise}(t)\\
				y(t) &= 
				\begin{bmatrix}
					1 & 0 & 0 & 0 
				\end{bmatrix}
				\begin{bmatrix}
					x_1 \\
					x_2 \\
					x_3 \\
					x_4 
				\end{bmatrix}(t)+\sqrt{R}\omega_{noise}(t)
		\end{split}}
		\label{eq11}
	\end{equation}
	We get a discretized nonlinear stochastic system in matrix form:
	\begin{equation}
		\boxed{\begin{split}
				x_{k+1} &= x_k + T_s
				\begin{bmatrix}
					-x_2 x_1 + x_3 v_{a,k} - x_4sign(x_1) \\
					0                                     \\
					0                                     \\
					0                                     
				\end{bmatrix} +\sqrt{T_s Q_d}v_{noise,k}\\
				y_k &= 
				\begin{bmatrix}
					1 & 0 & 0 & 0 
				\end{bmatrix}
				\begin{bmatrix}
					x_1 \\
					x_2 \\
					x_3 \\
					x_4 
				\end{bmatrix}_k+\sqrt{R}\omega_{noise,k}
		\end{split}}
		\label{eq12}
	\end{equation}
	\pagebreak
	\section{Identification using Extended Kalman Filter (EKF)}
	We have an EKF step below:\\
	\textbf{Initialize:}\\
	Select any
	\begin{itemize}
		\item {\makebox[1cm]{\(\hat{x}_{0|0}\)\hfill} initial state estimate}
		\item {\makebox[1cm]{\(P_{0|0}\)\hfill} positive definite error covariance matrix}
	\end{itemize}
	\textbf{Time Update}
	\begin{equation}
		\begin{split}
			\hat{x}_{k+1|k} &= f_d(\hat{x}_{k|k},u_k)\\
			P_{k+1|k} &= A_kP_{k|k}A^T_k+Q
		\end{split}
		\label{eq13}
	\end{equation}
	\textbf{Measurement Update}
	\begin{equation}
		\begin{split}
			\hat{y}_{k+1|k} &= h_d(\hat{x}_{k+1|k},u_{k+1})\\
			P_{xz,k+1|k}    &= P_{k+1|k}C^T_{k+1}\\
			P_{zz,k+1|k}    &= C_{k+1}P_{k+1|k}C^T_{k+1}+R\\
			\hat{x}_{k+1|k+1} &= \hat{x}_{k+1|k} + P_{xz,k+1|k}P^{-1}_{zz,k+1|k}(y_{k+1}-\hat{y}_{k+1|k})\\
			P_{k+1|k+1}     &= P_{k+1|k} - P_{xz,k+1|k}P^{-1}_{zz,k|k+1}P^T_{xz,k+1|k} 
		\end{split}
		\label{eq14}
	\end{equation}
	
	
	
	
	\subsection{Model with neglect the coulomb friction}
	From \autoref{eq7}, We have:
	\[\begin{split}
		x_{k+1} &= x_k + T_s
		\begin{bmatrix}
			-x_2 x_1 + x_3 v_{a,k} \\
			0                      \\
			0                      
		\end{bmatrix} +\sqrt{T_s Q_d}v_{noise,k}\\
		y_k &= 
		\begin{bmatrix}
			1 & 0 & 0 
		\end{bmatrix}
		\begin{bmatrix}
			x_1 \\
			x_2 \\
			x_3 
		\end{bmatrix}_k+\sqrt{R}\omega_{noise,k}
	\end{split}\]
	\begin{center}
		\textbf{Applying Extended Kalman Filter}
	\end{center}
	1. \textbf{Initialize a state and positive definite error covariance matrix}
	\[\begin{split}
		\hat{x}_{0|0} &= \begin{bmatrix}
			2 \\
			13 \\
			25
		\end{bmatrix} \text{ or some number randomly}\\
		P_{0|0} &= 2*eye(3) = 
		\begin{bmatrix}
			2 & 0 & 0 \\
			0 & 2 & 0 \\
			0 & 0 & 2 
		\end{bmatrix} \text{ or some number randomly}
	\end{split}\]
	
	2. \textbf{Compute state and covariance matrix in Time Update}\\
	\textbf{Initialize}
	\begin{itemize}
		\item {\makebox[2cm]{\(T_s = 0.01\)\hfill} sampling time (s) , up to user}
		\item \(Q = 0.00001
		\begin{bmatrix}
			10 & 0  & 0  \\
			0  & 25 & 0  \\
			0  & 0  & 25 
		\end{bmatrix} 
		= 0.00001 \times diag([10\quad 25\quad 25])\) \\process covariance matrix, smaller is truth in process model (use for tuning)
		\item \(R = 0.02
		\begin{bmatrix}
			1 & 0 & 0 \\
			0 & 1 & 0 \\
			0 & 0 & 1 
		\end{bmatrix} 
		= 0.02 \times diag([1\quad 1\quad 1])\) \\measurement covariance matrix, smaller is truth in measurement (use for tuning)
	\end{itemize}
	Compute \[\hat{x}_{k+1|k} = \hat{x}_k + T_s
	\begin{bmatrix}
		-x_2 x_1 + x_3 v_{a,k} \\
		0                      \\
		0                      
	\end{bmatrix} + \boxed{\sqrt{T_s Q_d}v_{noise,k}}\]
	\(\boxed{\sqrt{T_s Q_d}v_{noise,k}}\leftarrow\) put this bunch if we use in simulation to simulate noise to a true system, don't put if taking real value from system because the system has noise already.
	Compute
	\[P_{k+1|k} = A_kP_{k|k}A^T_k+Q\]
	where:
	\[A_k =
	\begin{bmatrix}
		-x_2 & -x_1 & v_a \\
		0    & 0    & 0   \\
		0    & 0    & 0   
	\end{bmatrix}\]
	is the jacobian matrix calculated by derivative the state model. From \autoref{eq4}
	\[J_f(x,y) =\begin{bmatrix}
		\frac{df_1}{dx_1} & \frac{df_1}{dx_2} & \frac{df_1}{dx_3} \\
		\frac{df_2}{dx_1} & \frac{df_2}{dx_2} & \frac{df_2}{dx_3} \\
		\frac{df_3}{dx_1} & \frac{df_3}{dx_2} & \frac{df_3}{dx_3}
	\end{bmatrix} = 
	\begin{bmatrix}
		\frac{-x_2x_1+x_3v_a}{dx_1} & \frac{-x_2x_1+x_3v_a}{dx_2} & \frac{-x_2x_1+x_3v_a}{dx_3} \\
		\frac{0}{dx_1}              & \frac{0}{dx_2}              & \frac{0}{dx_3}              \\
		\frac{0}{dx_1}              & \frac{0}{dx_2}              & \frac{0}{dx_3}              
	\end{bmatrix}
	\]
	3. \textbf{Compute state and covariance matrix in Measurement Update}\\
	Compute \(\hat{y}_{k+1|k} = h_d(\hat{x}_{k+1|k},u_{k+1})\) measurement estimation. This equation is the estimation of a measurement would look like. In our case, we measure the $\omega$ directly (\autoref{eq4}) and thus we can take:
	\[\hat{y}_{k+1|k} = [1\quad 0\quad 0]\hat{x}_{k+1|k} + D_ku_{a,k}\]
	\\
	Where: \(D_k = 0\)
	\\
	Compute \[
	\begin{split}
		P_{xz,k+1|k} &= P_{k+1|k}C^T_{k+1}\\
		P_{zz,k+1|k} &= C_{k+1}P_{k+1|k}C^T_{k+1}+R
	\end{split}\]
	From \autoref{eq4}, we have \(C = [1\quad 0\quad 0]\)
	\\
	Compute
	\[\boxed{\hat{x}_{k+1|k+1} = \hat{x}_{k+1|k} + P_{xz,k+1|k}P^{-1}_{zz,k+1|k}(y_{k+1}-\hat{y}_{k+1|k})}\]
	Where \(y_{k+1}\) is the measurement from sensor which has noise mixed inside. Get data directly from the sensor.
	\\
	Compute
	\[\boxed{P_{k+1|k+1} = P_{k+1|k} - P_{xz,k+1|k}P^{-1}_{zz,k|k+1}P^T_{xz,k+1|k}}\]
	
	
	
	
	\subsection{Model with coulomb friction}
	From \autoref{eq12}, We have:
	\[
	\begin{split}
		x_{k+1} &= x_k + T_s
		\begin{bmatrix}
			-x_2 x_1 + x_3 v_{a,k} - x_4sign(x_1) \\
			0                                     \\
			0                                     \\
			0                                     
		\end{bmatrix} +\sqrt{T_s Q_d}v_{noise,k}\\
		y_k &= 
		\begin{bmatrix}
			1 & 0 & 0 & 0 
		\end{bmatrix}
		\begin{bmatrix}
			x_1 \\
			x_2 \\
			x_3 \\
			x_4 
		\end{bmatrix}_k+\sqrt{R}\omega_{noise,k}
	\end{split}
	\]
	\begin{center}
		\textbf{Applying Extended Kalman Filter}
	\end{center}
	1. \textbf{Initialize a state and positive definite error covariance matrix}
	\[\begin{split}
		\hat{x}_{0|0} &= \begin{bmatrix}
			2 \\
			13 \\
			25 \\
			1
		\end{bmatrix} \text{ or some number randomly}\\
		P_{0|0} &= 2*eye(4) = 
		\begin{bmatrix}
			2 & 0 & 0 & 0 \\
			0 & 2 & 0 & 0 \\
			0 & 0 & 2 & 0 \\
			0 & 0 & 0 & 2 \\
		\end{bmatrix} \text{ or some number randomly}
	\end{split}\]
	
	2. \textbf{Compute state and covariance matrix in Time Update}\\
	\textbf{Initialize}
	\begin{itemize}
		\item {\makebox[2cm]{\(T_s = 0.01\)\hfill} sampling time (s) , up to user}
		\item \(Q = 0.00001
		\begin{bmatrix}
			10 & 0  & 0  & 0 \\
			0  & 25 & 0  & 0 \\
			0  & 0  & 25 & 0 \\
			0  & 0  & 0  & 1 
		\end{bmatrix} 
		= 0.00001 \times diag([10\quad 25\quad 25\quad 1])\) \\process covariance matrix, smaller is truth in process model (use for tuning)
		\item \(R = 0.02
		\begin{bmatrix}
			1 & 0 & 0 & 0 \\
			0 & 1 & 0 & 0 \\
			0 & 0 & 1 & 0 \\
			0 & 0 & 0 & 1 
		\end{bmatrix} 
		= 0.02 \times diag([1\quad 1\quad 1\quad 1])\) \\measurement covariance matrix, smaller is truth in measurement (use for tuning)
	\end{itemize}
	Compute \[\hat{x}_{k+1|k} = \hat{x}_k + T_s
	\begin{bmatrix}
		-x_2 x_1 + x_3 v_{a,k} - x_4sign(x_1) \\
		0                                     \\
		0                                     \\
		0                                     
	\end{bmatrix} + \boxed{\sqrt{T_s Q_d}v_{noise,k}}\]
	Compute
	\[P_{k+1|k} = A_kP_{k|k}A^T_k+Q\]
	where:
	\[A_k =
	\begin{bmatrix}
		-x_2 & -x_1 & v_a & -sign(x_1) \\
		0    & 0    & 0   & 0          \\
		0    & 0    & 0   & 0          \\
		0    & 0    & 0   & 0          
	\end{bmatrix}\]
	is the jacobian matrix calculated by derivative the state model from \autoref{eq11}.
	\\
	3. \textbf{Compute state and covariance matrix in Measurement Update}\\
	Compute
	\[\hat{y}_{k+1|k} = [1\quad 0\quad 0\quad 0]\hat{x}_{k+1|k} + D_ku_{a,k}\]
	\\
	Where: \(D_k = 0\)
	\\
	Compute \[
	\begin{split}
		P_{xz,k+1|k} &= P_{k+1|k}C^T_{k+1}\\
		P_{zz,k+1|k} &= C_{k+1}P_{k+1|k}C^T_{k+1}+R
	\end{split}\]
	From \autoref{eq11}, we have \(C = [1\quad 0\quad 0\quad 0]\)
	\\
	Compute
	\[\boxed{\hat{x}_{k+1|k+1} = \hat{x}_{k+1|k} + P_{xz,k+1|k}P^{-1}_{zz,k+1|k}(y_{k+1}-\hat{y}_{k+1|k})}\]
	Where \(y_{k+1}\) is the measurement from sensor which has noise mixed inside. Get data directly from the sensor.
	\\
	Compute
	\[\boxed{P_{k+1|k+1} = P_{k+1|k} - P_{xz,k+1|k}P^{-1}_{zz,k|k+1}P^T_{xz,k+1|k}}\]
	
	
	
	
	
	
	
	
	\begin{lstlisting}[language=MATLAB]
		function x_est = EKF(uk,y_true)
		Ts=0.01;
		persistent x_est_p P Qd_est R_est Qc_est;
		if isempty(x_est_p)
		x_est_p = [2;
		13;
		25;
		1];
		
		P = 2*[1 0 0 0;
		0 1 0 0;
		0 0 1 0;
		0 0 0 1;]                         %2*eye(4);
		
		Qc_est = 1e-5*[10  0   0  0;
		0  25   0  0;
		0   0  25  0;
		0   0   0  1;]               %1e-5*diag([10,25,25,1]);
		
		Qd_est=Qc_est*Ts;
		
		R_est=0.02;
		end
		
		c=[1 0 0 0];
		D=0;
		Ck=c;
		Dk=D;
		
		
		%Comput Kalman Gain and update predicted value
		Wk=P*Ck'/(Ck*P*Ck'+R_est);
		
		y_est=Ck*x_est_p+Dk*uk;
		
		x=x_est_p+Wk.*(y_true-y_est);
		
		%Compute prediction at next time step
		x_est=x+Ts*[-x(2)*x(1)+uk*x(3)-x(4)*sign(x(1));
		0                 ;
		0                 ;
		0                ];
		
		
		%Update error covariance matrices
		P=P-Wk*Ck*P;
		
		
		%Define Ak
		Ak=eye(4)+Ts*[-x(2) -x(1) uk -sign(x(1));
		0     0    0      0     ;
		0     0    0      0     ;
		0     0    0      0     ]; % jacobian
		
		P=Ak*P*Ak'+Qd_est;
		
		x_est_p=x_est;
	\end{lstlisting}
\end{document}